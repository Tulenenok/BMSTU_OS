\section{Виды атак на сетевом уровне}


% На сетевом уровне выделяют следующие атаки:

% \begin{enumerate}
%     \item микрофрагменты;
%     \item перекрытие пакетов;
%     \item IP Spoofing (подмена IP);
%     \item предсказание ТСР Sequence Number;
%     \item ARP Spoofing
%     \item затопление ICMP пакетами.
% \end{enumerate}

% \section{Виды атак на транспортном уровне}

% \begin{enumerate}
%     \item ранняя десинхронизация;
%     \item локальная буря;
%     \item затопление SYN пакетами (SYN flood).
% \end{enumerate}

% \subsubsection{Микрофрагменты}
% Атака позволяет установить соединении на уровне ТСР с узлом, находящимся под защитой пакетного фильтра, отбрасывающего ТСР пакеты с флагом SYN.

% Атака реализуется следующим образом: минимальный размер пакета IP, который можно передать без фрагментации, равен 68 байтам.

% Пакетные фильтры работают следующим образом: если они принимают фрагментированный IP пакет, то правила, установленные для 1-го фрагмента будут применяться по всем остальным пакетам по умолчанию.

% Следовательно, флаг SYN будет пропущен фильтром и остальные пакеты с флагом SYN. Пакет из фрагментов будет собран на узле жертвы и исполнен, тогда, с узлом жертвы будет установлено соединение на уровне ТСР.

% Данный вид атак классифицируется как атака на сетевом уровне.

% \subsubsection{Перекрытие пакетов}
% Правило фрагментации таково, что если последующий фрагмент по смещению перекрывает ранее полученный, он его перезаписывает поверх.

% 1-й фрагмент содержит нулевое значение флагов. 2-й – часть заголовка ТСР, начиная с Data offset с установленным флагом SYN, причем смещение фрагментов в заголовке IPуказывает, что смещение начинается с 13 байта.

% 1-й пакет пропускается, 2-й поверх 1-го, а при сборе формируется SYN, тогда с узлом жертвы образуется соединение на уровне ТСР.

% Данный вид атак классифицируется как атака на сетевом уровне.

% \subsubsection{IP Spoofing (подмена IP)}
% Целью данной атаки является посылка пакетов, содержащих чужой адрес отправителя с целью сокрытия источника атаки, либо с целью извлечь выгоду из доверительно связи двух узлов, использующих в качестве идентификатора IP адреса.

% Данный вид атак классифицируется как атака на сетевом уровне.

% \subsubsection{Ранняя десинхронизация}
% Ранняя десинхронизация --- атака, направленная на нарушение доступности при соединении клиентсервер.

% Начальные условия: нарушитель находится на пути трафика от клиента (К) к серверу (С), имеет возможность его прослушивать и выполнять роль его посредника, генерирующего корректные пакеты для К и С.

% \begin{enumerate}
%     \item К отправляет запрос на установку соединения;
%     \item С отвечает. К считает соединение установленным;
%     \item Н отправляет серверу RST (сброс);
%     \item От имени К Н отсылает запрос на установление соединения с новым значением ASN;
%     \item С возвращает К подтверждение установления соединения;
%     \item Н от имени К подтверждает установление соединения.
% \end{enumerate}

% В результате атаки у К и С соединения считается установленным, однако, оно десинхронное из-за того, что у них не совпадают начальные порядковые номера сегментов протокола ТСР, т.е. любой пакет в рамках соединения будет не попадать в диапазон окна, т.е. будет считаться запрещенным. В ответ на него будет сформирован пакет переспроса, который будет запрещен для другой стороны, т.е. такое соединение спровоцирует «АСК - бурю». Узлы будут постоянно обмениваться запрещенными пакетами АСК. Это будет продолжаться до тех пор пока один из пакетов не будет утерян.

% Данный вид атак классифицируется как атака на транспортном уровне.

% \subsubsection{Локальная буря}
% Порт 7 --- echo, любой пакет возвращается обратно, порт 19 --- chargen, возвращает пакет со строкой символов.

% Нарушитель посылает единственный пакет UDP, где в качестве исходного порта указан 7, а в качестве порта получателя 19, а в качестве адресов, либо адрес двух атакуемых машин локальной сети, либо адрес одной и той же машины. В результате, попав на порт 19, пакет будет отработан и в ответ послана строка на порт 7, а 7 отразит ее на 19. Бесконечный цикл добавляет нагрузку канала и машины.

% Данный вид атак классифицируется как атака на транспортном уровне.

% \subsubsection{Затопление SYN пакетами (SYN flood)}
% SYN-запрос на установление соединения. В ответ на пакет SYN узел отвечает пакетом SYN ASK и переводит соединение из состояния Listen в состоянии SYN-Received. Соединение в этом состоянии переводится в специальную очередь соединений, ожидающих установления. У большинства узлов эта очередь конечна и при ее заполнении все следующие соединения будут отброшены, т.е. нарушителю достаточно генерировать не очень большое количество SYN запросов, не отвечая на подтверждения, чтобы полностью исключить возможность подключения к узлу легальных пользователей.

% Данный вид атак классифицируется как атака на транспортном уровне.