\chapter{Системный вызов open()}

Системный вызов open() открывает файл, указанный в pathname. Если указанный файл не существует, он может (необязательно) (если указан флаг O\_CREATE) был создан open().

\begin{center}
\captionsetup{justification=raggedright,singlelinecheck=off}
\begin{lstlisting}[label=lst:add,caption=Системный вызов open()]
    #include <sys/types.h>
    #include <sys/stat.h>
    #include <fcntl.h>

    int open(const char *pathname, int flags);
    int open(const char *pathname, int flags, mode_t mode);
\end{lstlisting}
\end{center}

Возвращаемое значение open() --- дескриптор файла, неотрицательное целое число, которое используется в последующих системных вызовах для работы с файлом. 

Параметр pathname --- это имя файла в файловой системе: полный путь к файлу или сокращенное имя.

Параметр flags --- это режим открытия файла, представляющий собой
один или несколько флагов открытия, объединенных оператором побитового ИЛИ. Список доступных флагов:

\begin{itemize}
    \item[---] O\_EXEC --- открыть только для выполнения (результат не определен, при открытии директории).
    \item[---] O\_RDONLY --- открыть только на чтение.
    \item[---] O\_RDWR --- открыть на чтение и запись.
    \item[---] O\_SEARCH --- открыть директорию только для поиска (результат не определен, при использовании с файлами, не являющимися директорией).
    \item[---] O\_WRONLY --- открыть только на запись.
    \item[---] O\_APPEND --- файл открывается в режиме добавления, перед каждой операцией записи файловый указатель будет устанавливаться в конец файла.
    \item[---] O\_CLOEXEC --- включает флаг close-on-exec для нового файлового дескриптора, указание этого флага позволяет программе избегать дополнительных операций fcntl F\_SETFD для установки флага FD\_CLOEXEC.
    \item[---] O\_CREAT --- если файл не существует, то он будет создан.
    \item[---] O\_DIRECTORY --- если файл не является каталогом, то open вернёт ошибку.
    \item[---] O\_DSYNC --- файл открывается в режиме синхронного ввода-вывода (все операции записи для соответствующего дескриптора файла блокируют вызывающий процесс до тех пор, пока данные не будут физически записаны).
    \item[---] O\_EXCL --- если используется совместно с O\_CREAT, то при наличии уже созданного файла вызов завершится ошибкой.
    \item[---] O\_NOCTTY --- если файл указывает на терминальное устройство, то оно не станет терминалом управления процесса, даже при его отсутствии.
    \item[---] O\_NOFOLLOW --- если файл является символической ссылкой, то open вернёт ошибку.
    \item[---] O\_NONBLOCK --- файл открывается, по возможности, в режиме non-blocking,
то есть никакие последующие операции над дескриптором файла не заставляют в дальнейшем вызывающий процесс ждать.
    \item[---] O\_RSYNC --- операции записи должны выполняться на том же уровне, что и O\_SYNC.
    \item[---] O\_SYNC --- файл открывается в режиме синхронного ввода-вывода (все операции записи для соответствующего дескриптора файла блокируют вызывающий процесс до тех пор, пока данные не будут физически записаны).
    \item[---] O\_TRUNC --- если файл уже существует, он является обычным файлом и заданный режим позволяет записывать в этот файл, то его длина будет урезана до нуля.
    \item[---] O\_LARGEFILE --- позволяет открывать файлы, размер которых не может быть представлен типом off\_t (long).
    \item[---] O\_TMPFILE --- при наличии данного флага создаётся неименованный временный файл.
\end{itemize}

Параметр mode всегда должен быть указан при использовании O\_CREATE; во всех остальных случаях этот параметр игнорируется.

